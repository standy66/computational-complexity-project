\documentclass{article}
\usepackage[english,russian]{babel}
\usepackage[utf8]{inputenc}
\usepackage{amssymb}
\usepackage{amsfonts}
\usepackage{amsmath}
\usepackage{hyperref}
\usepackage{amsmath}
\usepackage{amsthm}
\usepackage{listings}
\usepackage{color}

\definecolor{mygreen}{rgb}{0,0.6,0}
\definecolor{mygray}{rgb}{0.95,0.95,0.95}
\definecolor{mymauve}{rgb}{0.58,0,0.82}

\lstset{ %
  backgroundcolor=\color{mygray},   
  basicstyle=\small\ttfamily,  
  breaklines=true,                 
  captionpos=b,                    
  commentstyle=\color{mygreen},    
  escapeinside={\%*}{*)},          
  keywordstyle=\color{blue},       
  stringstyle=\color{mymauve},
  numbers=left,
  numbersep=5pt,
  numberstyle=\small     
}

\newtheorem{theorem}{Теорема}

\title{Задача $\mathsf{SETCOVER}$}
\author{Андрей Степанов}

\begin{document}
\maketitle

\section*{Формулировка задачи}
Сформулируем задачу о покрытии множествами $\mathsf{SETCOVER}$. Пусть дано конечное множество $\mathcal{U}$, называемое вселенной и конечная система его подмножеств $\mathcal{F} = \{S_1, \ldots, S_n\}$. Без ограничения общности можно считать, что $\mathcal{U} = \{1, \ldots, m\}$. В задаче поиска требуется найти минимальное по мощности покрытие $\mathcal{G} \subset \mathcal{F}$, такое что
$$\mathcal{U} = \bigcup_{S \in \mathcal{G}} S$$
В задаче распознавания требуется проверить, если ли такое покрытие мощности $\leq k$.

\section*{NP-полнота задачи $\mathsf{SETCOVER}$}
Сначала напомним формулировку задачи $\mathsf{VERTEXCOVER}$. Пусть дан неориентированный граф $G = (V, E)$ и число $k$. Требуется проверить, есть ли в графе вершинное покрытие размера $\leq k$. Вершинным покрытием называется такое $V' \subset V$, что для любого ребра $\{u, v\} \in E$ выполнено, что  $u \in V'$ или $v \in V'$. Известно, что данная задача является $\mathbf{NP}$-полной \footnote{Это показано, например в [1]}.

\begin{theorem}
Задача $\mathsf{SETCOVER}$ является $\mathbf{NP}$-полной.
\end{theorem}

\begin{proof}Докажем сначала $\mathbf{NP}$-трудность задачи $\mathsf{SETCOVER}$, сведя к ней $\mathbf{NP}$-полную задачу $\mathsf{VERTEXCOVER}$.

Пусть дан граф $G = (V, E)$ и число k. Пусть $|V| = n, |E| = m$. Положим $\mathcal{U} = E$, $\mathcal{F} = \{A_v : v \in V\}$, где $A_v = \{e \in E: v \in e\}$ -- множество рёбер, покрываемых вершиной $v$. Ясно, что такая  сводимость является полиномиальной, чтобы построить множество $A_v$, достаточно пройтись по всем рёбрам графа, которых $m$. Значит, чтобы построить $\mathcal{F}$ нужно $O(nm)$ времени, а на построение $\mathcal{U}$ нужно $O(m)$ времени. Суммарно это $O(|G|)$. Докажем, что эта сводимость корректна.

Пусть в графе $G$ нашлось вершинное покрытие $V', |V'| \leq k$. Тогда $\mathcal{G} = \{A_v: v \in V'\}$ будет покрытием множества $\mathcal{U}$ (по определению вершинного покрытия $V'$) мощности $\leq k$ (так как $|\mathcal{G}| = |V'| \leq k$).

Наоборот, пусть нашлось покрытие $\mathcal{G} = \{A_v: v \in V'\}$ множества $\mathcal{U}$ размера $\leq k$. Тогда $|V'| = |\mathcal{G}| \leq k$ и вершины из $V'$ покрывают все рёбра графа $G$ по определению покрытия $\mathcal{G}$.

Итак, $\mathsf{VERTEXCOVER} \leq_p \mathsf{SETCOVER}$. Осталось показать, что $\mathsf{SETCOVER} \in \mathbf{NP}$. Действительно, для задачи $\mathsf{SETCOVER}$ легко построить верификатор $V(x, s)$. Сертификатом будет служить само покрытие. Действительно, покрытие является подмножеством входа программы, то есть сертификат имеет полиномиальную длину от размера входа. Задачей верификатора будет проверить, действительно ли $s$ является покрытием вселенной $\mathcal{U}$. Для этого нужно для каждого элемента вселенной проверить, лежит ли он в каком-либо множестве в покрытии. Это можно сделать за полиномиальное время.
\end{proof}

\section*{Жадный алгоритм и $\log n$ приближение}
Опишем жадный алгоритм GREEDY решения задачи о покрытии. В каждый момент времени будем выбирать множество $Ф A \in \mathcal{F}$, которое покрывает как можно больше ещё не покрытых элементов в множестве $\mathcal{U}$, и добавлять множество $A$ в покрытие. Поскольку этот алгоритм пытается максимизировать покрытие только на данном шаге, он называется "жадным". Так как "не смотрит в будущее".

Приведём реализацию этого алгоритма на языке \textsf{Python}:
\begin{lstlisting}[language=python]
def greedy(universe, sets)
    set_cover = []
    while (universe):
        best_set = []
        best_set_coverage = 0
        for s in sets:
            coverage = 0
            for elem in s:
                if (elem in universe):
                    coverage = coverage + 1
            if (coverage >= best_set_coverage):
                best_set = s
        if (best_set == []):
            raise RuntimeError("no solution")
        for elem in best_set:
            if (elem in universe):
                universe.remove(elem)
        sets.remove(best_set)
        set_cover.append(best_set)
    return set_cover

\end{lstlisting}
Оказывается, жадный алгоритм даёт $\log n$-приближение, в смысле, который сформулирован в следующей теореме.

\begin{theorem}
Пусть $OPT$ -- оптимальный ответ для задачи \textsf{SETCOVER} на входе $(\mathcal{U}, \mathcal{F})$. Обозначим $n = |\mathcal{U}|$. Тогда
алгоритм GREEDY запущенный на входе $(\mathcal{U}, \mathcal{F})$ выдаст ответ, по размеру не превосходящий $\log n * |OPT|$.
\end{theorem}

\begin{proof}
Пусть $\mathcal{A} = \{A_1, \ldots, A_k\} \subset \mathcal{F}$ -- ответ, выданный алгоритмом GREEDY, причём множества $A_i$ пронумерованы в том порядке, в котором их выбирал алгоритм. Так как оптимальное решение использует $|OPT|$ множеств, то по принципу Дирихле, есть множество в $\mathcal{F}$, которое покрывает хотя бы $\frac{n}{|OPT|}$ точек вселенной $\mathcal{U}$. Поскольку жадный алгоритм выбирает каждый раз множество, которое покрывает как можно больше точек, то $A_1$ покрывает хотя бы $\frac{n}{|OPT|}$. Значит ещё не покрытых точек осталось не больше $n \Big(1 - \frac{1}{|OPT|}\Big)$. Далее, либо $A_1 \notin OPT$, либо $A_1 \in OPT$. В первом случае в $OPT$, а следовательно и в $\mathcal{F}$ всё ещё найдется по принципу Дирихле множество, покрывающее $\frac{1}{|OPT|}$-долю оставшихся точек. Во втором случае по тому же самому принципу Дирихле найдется множество в $OPT \setminus \{A_1\}$ покрывающее $\frac{1}{|OPT| - 1}$-долю оставшихся точек, а значит и $\frac{1}{|OPT|}$-долю. В любом случае, в силу природы жадного алгоритма, $A_2$ покрывает хотя бы $\frac{1}{|OPT|}\Big(n - \frac{1}{|OPT}\Big)$ точек. Значит осталось $n \Big(1 - \frac{1}{|OPT|} \Big)^2$ точек. Тогда после $|OPT| \log n$ шагов по индукции получаем, что осталось 
$$n \Big(1 - \frac{1}{|OPT|}\Big)^{|OPT| \log n} < n \Big(\frac{1}{e}\Big)^{\log n} = 1$$ точек, а значит, к этому времени алгоритм завершился и $k < |OPT| \log n$.
\end{proof}

\section*{Частные случаи и k-приближение}
Рассмотрим следующий частный случай задачи $\mathsf{SETCOVER}$. Пусть каждый элемент вселенной $\mathcal{U}$ присутствует не более чем в $k$ подмножествах семейства $\mathcal{F}$. Оказывается, в этом случае можно построить полиномиальный алгоритм, дающий $k$-приближение, сведя задачу к задаче о линейном программировании (ЛП)

Действительно, сначала сведем задачу к задаче о целочисленном линейном программировании (ЦЛП). Пусть $x_i$ -- индикатор того, что $S_i$ множество входит в покрытие. Тогда задача \textsf{SETCOVER} формулируется как задача ЦЛП в следующем виде

\begin{equation}
\label{ilp}
\begin{cases}
\min x_1 + \ldots + x_n \\
\forall i: x_i \in \{0, 1\} \\
\forall u \in \mathcal{U}: \displaystyle \sum_{j: u \in S_j} x_j \geq 1
\end{cases}
\end{equation}

Заметим, что в каждой сумме
$$\Sigma_u = \sum_{j: u \in S_j} x_j$$
не более чем $k$ слагаемых. Значит, если $\Sigma_u \geq 1$, то найдется $x_j \geq 1/k$, даже если отсутствуют ограничения на целочисленность переменных $x_j$. Это даёт нам право сформулировать следующей алгоритм LPSETCOVER.

Решшим задачу ЛП \ref{ilp} без ограничения на целочисленность за полиномиальное время \footnote{Алгоритм приведён в [2]}, то есть ту, где все $x_i \geq 0$. В каждой сумме есть хотя бы один $x_j >= 1/k$, округлим все такие $x_j$ до 1, а все остальные положим равными $0$. Понятно, что полученный набор $(x_1, \ldots, x_n)$ задает покрытие множества $\mathcal{U}$ в силу вышенаписанного утверждения. 

Осталось понять, что в процессе округления мы каждый $x_j$ из решения ЛП умножили на некоторое число, не превосходящее $k$, чтобы получить 0 или 1. Значит, верно следующее
$$|LPSETCOVER| \leq k * |LPOPT| \leq k * |ILPOPT| = k * |OPT|$$, где $LPSETCOVER$ -- ответ, выданный алгоритмом, $LPOPT$, $ILPOPT$, $OPT$ -- оптимальные ответы для задач линейного программирования, целочисленного линейного программирование и задачи \textsf{SETCOVER} соответственно. В


\begin{thebibliography}{9}
\bibitem{musatych}
D. Musatov
\textit{Lecture notes on computational complexity}
\\\texttt{http://ru.discrete-mathematics.org/fall2015/3/complexity/lecture-3-4-np-complete.pdf}
 
\bibitem{karmarkar} 
Robert J. Vanderbei, Marc S. Meketon, Barry A. Freedman
\textit{A Modification of Karmarkar's Linear Programming Algorithm}
\\\texttt{http://www.princeton.edu/~rvdb/tex/myPapers/VanderbeiMeketonFreedman.pdf}

\bibitem{avrim} 
Avrim Blum
\textit{Lecture notes on computer science}
\\\texttt{http://www.cs.cmu.edu/~avrim/451f12/lectures/lect1106.pdf}
\end{thebibliography}
\end{document}